\documentclass[12pt,titlepage]{article}

% Packages
\usepackage[letterpaper, margin=1in]{geometry}
\usepackage{fancyhdr}
\usepackage{hyperref}

% Header/Footer
\pagestyle{fancy}
\fancyhf{}
\fancyhead[L]{Software Requirement Specification}
\fancyhead[R]{\today}
\fancyfoot[C]{\thepage}

% Title
\title{UnBox3D: \\ Software Requirement Specification}
\author{Version \\ 2.0.0}
\date{Group 1: Alexander Ramirez, Vivian Cases, Jacky Lim, \\ Nicholas Sisneros, Vince Wang}

\begin{document}

\maketitle
\thispagestyle{empty}

\newpage

\tableofcontents

\newpage

\section*{Version Description}
\addcontentsline{toc}{section}{Version Description}
\begin{center}
\begin{tabular}{|c|c|c|c|}
\hline
User & Date & Description & Version \\
\hline
Vivian Casas & \today & Update for snapshot 1 & 1.0 \\
\hline
Vivian Casas & \today & Update for snapshot 2 & 2.0 \\
\hline
Alexander Ramirez & \today & Update for snapshot 3 & 3.0 \\
\hline
\end{tabular}
\end{center}


\section{Introduction}
\subsection{Purpose}
The purpose of this document is to provide a detailed description of the UnBox3D application. Version 3 adds onto the two previous checkpoints. In this checkpoint, we aim to introduce model simplification. This is achieved in two ways: removing objects based on size and simplifying per-object with decimation. This document outlines the system architecture, core components, user interfaces, and design considerations that support the conversion of complex 3D models into simplified geometric representations and corresponding 2D layouts for fabrication. This document serves as a foundation for understanding the requirement specifications and implementation strategies behind UnBox3D.
\subsection{Intended Audience}
This document is intended for software developers, project managers, and stakeholders involved in the design, development, and deployment of UnBox3D. It may also be of interest to engineers and defense personnel who will use the application to streamline prototyping and fabrication processes.
\subsection{Overview of the Software}
The UnBox3D application is a software tool that bridges the gap between digital modeling and physical assembly. In Version 2, the application now supports importing and rendering 3D models in .obj format, allowing users to visualize their models within the program by parsing the data using Assimp and rendering it with OpenGL via OpenTK. The application simplifies 3D models into basic geometric shapes and flattens them into printable 2D patterns. These layouts can then be used for rapid prototyping and fabrication. The application will provide a user-friendly interface and is designed to improve efficiency, accuracy, and accessibility in engineering and defense applications.
\subsection{User Interface}
\begin{itemize}
    \item \textbf{Splash Screen:} The logo is shown as the program loads on the monitor
    \item \textbf{Import:} The import button is now functional, allowing users to either drop a .obj file into the viewport or click to open a file dialog to select the model they wish to import.
    \item \textbf{Rendering Screen:} A viewport where the imported 3D object can be viewed and rendered using OpenGL.
    \item \textbf{Object Hierarchy:} A list of all the objects within the imported 3D model. Here, we can select the object we want to simplify.
    \item \textbf{Simplify:} A button that triggers the simplification process on the currently selected object.
    \item \textbf{Simplify Slider:} A slider that allows removing objects from the scene if they are smaller than the size threshold specified.
    \item \textbf{Export:} Export button remains as a placeholder for future functionality.
\end{itemize}
\subsection{Software Interfaces}
\begin{itemize}
    \item OpenGL is used to render imported 3D models.
    \item Assimp will be used in order to parse the data from the imported models
    \item  OpenTK will be used as a wrapper so that we can work on OpenGL in C\#
    \item Blender Python API to unfold the model
\end{itemize}

\section{Legal and Ethical Considerations}
\subsection{Data Storage and Privacy}
At this stage of development, UnBox3D does not collect or store user data. The initial checkpoint focuses on establishing the base program structure, including interface placeholders and backend setup. As the project progresses, any future implementation involving user-generated content, model uploads, or fabrication data will require secure storage practices. If persistent data storage is introduced, the application will comply with relevant data protection regulations such as the General Data Protection Regulation (GDPR) and the California Consumer Privacy Act (CCPA). Users will be informed of any data collection and given options to manage their privacy preferences.
\subsection{Legal/Ethical Issues}
Unbox3D is a general-purpose tool for converting 3D models into 2D unfolded layouts for
fabrication. Although the initial sponsor is a U.S. Army research organization and one intended
use is the generation of mock training targets, the software is not limited to military applications.
It can also support civilian, educational, and commercial use cases such as architectural models,
prototyping, hobbyist projects, and training aids. The development team therefore strives to keep
the software functionally neutral and broadly applicable, minimizing direct coupling to any
single military workflow.

From an ethical standpoint, contributors should be aware that any tool capable of improving
fabrication efficiency in a military context may indirectly support armed conflict. At the same
time, the tool can enable safer training, reduce material waste, and provide value to non-military
users. The project is considered ethically acceptable under the assumption that it remains
general-purpose, does not incorporate offensive or targeting logic, and is documented in a way
that encourages peaceful and educational applications in addition to the sponsor's needs.

Legally, Unbox3D is required to be built entirely on free and open-source software (FOSS)
components. The final project license is expected to be a permissive or copyleft FOSS license
(e.g., GPLv3 or a BSD-style license), ensuring that recipients retain the right to inspect, modify,
and redistribute the source code in accordance with the chosen license terms. All third-party
libraries and tools (including Blender and OpenTK) must be used in compliance with their
respective licenses, and no proprietary or paid dependencies are to be introduced.

Users are responsible for ensuring that any 3D models processed with Unbox3D respect
intellectual property rights, export-control regulations, and organizational classification rules.
The project does not grant any rights to redistribute imported models; it only provides a
mechanism to transform models that the user is already authorized to use.


\section{Glossary}
\begin{tabular}{|c|c|}
\hline
Acronym & Definition \\
\hline
GDPR & General Data Protection Regulation \\
CCPA & California Consumer Privacy Act \\
Assimp & Open Asset Import Library \\
OpenTK & Open Toolkit Library \\
\hline
\end{tabular}

\end{document}
